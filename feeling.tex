\newpage
\section*{Impressions} % (1 pages) Impressions personnelles
\subsection{Journée type et évènements.}
La journée type commence à 9h avec un café, elle est ponctuée d'une pause déjeuner de 12h à 14H et se finit à 18h.\\
Fréquemment, des évènements sont organisés par l'entreprise, ou de manière plus informelle entre collègues.
Le vendredi matin, il y a la tradition du petit déjeuner : chaque semaine un salarié ramène les croissants pour toute l’entreprise. Le vendredi midi, il arrive souvent que certains aillent au restaurant. 
Enfin, une fois par mois, un afterwork est organisé.

\subsection{Difficultés rencontrées}
\paragraph*{L’architecture de l’application\\}
Il m’a fallu un certain temps pour appréhender l’architecture de l’application et comprendre le flot d’exécution d’une requête.
J’ai parfois eu du mal à savoir quelles classes, composants et services je devais implémenter et à quel moment chacun intervenait. Durant l'implémentation de ma première fonctionnalité, mon collègue Abdessalam a souvent été là pour me guider.

\paragraph*{Maîtrise de nouvelles technologies\\}
Lorsque j’ai commencé mon stage je ne connaissais pas la plupart des technologies utilisées sur le projet. L'utilisation d'hibernate, Spring, Jsf et primefaces était nouvelle pour moi. Cependant, je connaissais des technologies proches de celles utilisées : java, j2ee, jsp et j'ai donc pu faire des parallèles avec mes connaissances (issues du monde java ou autres).

\paragraph*{Impression générale\\}
Après ces six semaines dans l'entreprise, mes premières impressions sont trés bonnes. \\
Je trouve que les aspects métiers liés au projet sont intéressants, et il est motivant de savoir que le logiciel sera amené dans le futur à être utilisé par de nombreux cabinets médicaux.\\
Ensuite, il est appréciable de travailler sur un projet relativement récent (environ un an). Des efforts importants sont portés sur la conception et l'architecture de la solution. Le code est donc clair, maintenable et il est facile de rajouter des fonctionnalités.\\ 
Ensuite, j'apprécie les récentes démarches portant sur l'amélioration en continue des outils de production (mise en place du serveur d'intégration continue Jenkins, de l'inspecteur de qualité de code SonarQube et du gestionnaire de dossiers Nexus).\\
Pour finir, j'apprécie le fait de pouvoir développer mes connaissances et mes compétences en java 8, Hibernate, Spring et JSF.\\

En ce qui concerne l'ambiance de travail, elle est agréable. Les déjeuners entre collègues, les afterworks et autres évènements sont très appréciables.