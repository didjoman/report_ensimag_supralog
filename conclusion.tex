\newpage
\section*{Conclusion} % (1 pages) Impressions personnelles

% Travail réalisé
Le but de mon projet de fin d'études était d'ajouter à Topaze Web le dossier médical d'un patient en me basant sur celui implémenté dans Topaze Maestro.\\
Le dossier médical d'un patient est une fonctionnalité composite. Elle est constituée d'une liste de documents et de fonctionnalités annexes telles que le gestion des documents texte ou la numérisation de documents.\\

% Complexité confrontées
La réalisation du projet était intéressante, car elle a présenté différents enjeux. Tout d'abord, avant de pouvoir développer, il était nécessaire de comprendre l'architecture mise en place, le modèle de données de l'application, les conventions propres au projet et le générateur de code. Par la suite, j'ai du me familiariser avec les technologies qui étaient nouvelles pour moi (Hibernate, JSF). Enfin, certaines fonctionnalités ont nécessité une attention particulière lors de la conception ou l'implémentation (champs génériques dans l'éditeur de texte, introduction des postes utilisateurs etc.)\\
% Contraintes du projet
%    Défis : comprendre les aspects métiers liés au logiciel à implémenter. 
%    REjoindre un projet d'un an.
%    Adaptation aux méthodes de travail de l'équipe
%    Apprentissage des techonlogies utilisées sur le projet
%    Proposer des solutions techniques capables de résoudre les problématiques posées.
%    Topaze Maestro respect existant, innovation
%    Contraintes : architecture, qualité de code, maintenabilité, durabilité.

% Apports techno
Le fait d'avoir développé entièrement la fonctionnalité du dossier médical a été une expérience très enrichissante. En effet, cela m'a permis de travailler à la fois sur le front-end et le back-end de l'application et ainsi de développer mes compétences sur les technologies HTML5, CSS3, Javascript, Java 8, J2EE, JSF, Spring MVC et Hibernate. \\
Chacune des fonctionnalités implémentée présentait un intérêt particulier.\\
L'implémentation de la liste des documents m'a permis de découvrir l'architecture multi-tiers utilisée et d'appréhender les différentes technologies.\\ L'intégration et l'extension de l'éditeur de texte Ckeditor m'a permis de découvrir la communication entre Java et Javascript, les composants JSF et l'introspection de Java.\\
La fonctionnalité de numérisation m'a permis de travailler sur le serveur Sésame et le driver Pyvital. J'ai ainsi pu manipuler les websockets, développer en python et découvrir les librairies Twain et Pillow.\\
% Technos apprises
%   Archi n-tiers, communication entre serveurs et drivers
%   Génération de code
%   Python avec driver
%   Websockets
%   Spring + injection + aspects + JSF + J2EE + JAVA 8 + HIBERNATE
%	Cookies, programmation par aspects, introspection de java 

Au delà des aspects techniques, le stage m'a permis de travailler chez un éditeur de logiciel, dans une entreprise de petite taille (PME). J'ai ainsi pu faire la comparaison avec mon stage précédent, chez Atos.\\
En ce qui concerne le projet, il m'a permis de travailler dans une équipe qui applique la méthode agile Scrum et pratique les revues code. D'autre part, j'ai pu découvrir les aspects métiers liés à la sécurité sociale, aux mutuelles ou à la facturation.\\
%En ce qui concerne l'ambiance de travail, elle est agréable. Les déjeuners entre collègues, les afterworks et autres évènements sont très appréciables.

Pour conclure, le stage était très enrichissant car il m'a permis de travailler dans un environnement java, de faire du développement fullstack, de découvrir les aspects métiers liés à l'e-santé, de travailler en mode agile et d'apprécier le travail chez un éditeur logiciel.
% Intéret du projet
% 	Communication Java - js
% 	Récent,
% 	Architecture
% 	Aspects métiers
%   Gain méthodologies ()Méthode de travail agile)

% Contraintes du projet
%    Défis : comprendre les aspects métiers liés au logiciel à implémenter. 
%    REjoindre un projet d'un an.
%    Adaptation aux méthodes de travail de l'équipe
%    Apprentissage des techonlogies utilisées sur le projet
%    Proposer des solutions techniques capables de résoudre les problématiques posées.
%    Topaze Maestro respect existant, innovation
%    Contraintes : architecture, qualité de code, maintenabilité, durabilité.


% Entreprise