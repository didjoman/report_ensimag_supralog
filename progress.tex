\newpage

%TODO Si travail en mode agile, inclure une analyse de la méthode de travail, (identifier ce qui était prévu au départ au niveau macro, l'implémentation effective, les virages pris après chaque sprint après les retours clients).

\section{Planning prévisionnel et réalisé} % (2 pages) Diagramme de gant avec planning prévisionnel
\subsection{Sprint 2}
Voici le diagramme de Gantt prévisionnel : 

\begin{figure}[H]
  \centering
  \includegraphics[width=17cm]{./img/gantt_sprint_2}
  \caption{\label{fig:mb_va_ast} Gantt prévisionnel et Gantt réalisé du Sprint 2.}
\end{figure}

Et voici le diagramme de gantt réel : 



\subsection{Sprint 3}
\begin{figure}[H]
  \centering
  \includegraphics[width=17cm]{./img/gantt_sprint_3}
  \caption{\label{fig:mb_va_ast} Gantt prévisionnel et Gantt réalisé du Sprint 3.}
\end{figure}

\subsection{Sprint 4}

\begin{figure}[H]
  \centering
  \includegraphics[width=17cm]{./img/gantt_sprint_4}
  \caption{\label{fig:mb_va_ast} Gantt prévisionnel et Gantt réalisé du Sprint 4.}
\end{figure}

%TODO Planning prévisionnel 
%TODO Planning réalisé + explications

%TODO définition des tâches 
%TODO Différences prévisionnel/réalisé
%TODO Tâches particulières

\section{Outils d'assistance au développement utilisés.}
Pour réaliser la gestion du projet, nous utilisons l'outil \textit{Jira}.
Jira permet de découper les fonctionnalités à réaliser en tâches et de les affecter ensuite aux membres du projet.\\
Au fur et à mesure que le projet avance, chaque développeur journalise le nombre d'heures qu'il a travaillé sur chaque tâche et fait évoluer leur statut (tâche ouverte, en cours, en cours de revue, finie).\\
L'utilité de Jira est donc double puisqu'il permet à la fois de planifier le projet, mais également de suivre son avancement. La maitrise d'oeuvre s'en sert donc pour s'organiser et la maitrise d'ouvrage l'utilise pour se tenir au courant de l'avancement des travaux.\\

Le deuxième outil utilisé par l'équipe est \textit{Confluence}. Il permet de partager et gérer les documents propres au projet. De plus, Il offre une gestion de version et l'aperçu instantané des documents. On l'utilise pour stocker la documentation fonctionnelle, la documentation technique, l'architecture du projet, et le manuel utilisateur.\\

Enfin, pour l'hébergement du code du projet, nous utilisons \textit{Bitbucket} associé à\textit{Git}. \textit{Bitbucket} permet d'héberger le projet sur un répertoire privé. \textit{Git} permet de gérer les différentes versions du projet et de mettre en commun le travail réalisé par les développeurs. 

%TODO Outils d'assistance au développement utilisés.

\section{Méthode de travail employée}
La méthode de travail utilisée est proche de la méthode Scrum.
\paragraph*{Les sprints}
Le projet est découpé en "Sprints". Un sprint est une période de temps donnée, durant laquelle un certain nombre de fonctionnalités doivent être implémentées par les développeurs. A la fin de la période, le travail réalisé est présenté au client.\\

La méthode Scrum recommande des sprints d'une durée d'environ deux semaines. Cependant, dans le cadre de notre projet, les sprint ont plutôt une durée comprise entre un et deux mois. Cela se justifie notamment par le fait que certaines fonctionnalités peuvent avoir un temps de développement long.\\
Le fait que la durée soit plus longue que celle recommandée présente des intérêts et des désavantages. 
Un sprint plus long permet de solliciter le client moins souvent, de limiter le temps alloué à la gestion du sprint lui même, de limiter le nombre de rendus.
Le point négatif d'une durée allongée est que cela limite le nombre de retours clients et augmente ainsi le risque d'une inadéquation entre le travail réalisé et l'attendu.

\paragraph*{Les stand-up meetings}
La méthode Scrum introduit le concept de "Mêlée quotidienne" (aussi appelé "daily scrum" ou "stand-up meeting").\\
D'après les recommandations de la méthode, ce type de réunion doit être de courte durée (15 minutes), se dérouler de préférence le matin et aborder trois questions :
\begin{sitemize}
\item Quel travail a été effectué la veille ?
\item Quel travail sera réalisé dans la journée ?
\item Y a-t-il des obstacles qui empêchent la réalisation des objectifs fixés ?
\end{sitemize}

Dans le cadre du projet, des stand-up meetings sont réalisés à raison d'une fois par semaine environ. Ces réunions permettent à l'équipe d'être au courant de ce que chacun fait et n'ont pas pour but de résoudre des problèmes.\\
Lorsqu'un problème se pose, il est généralement discuté directement entre le chef de projet et le développeur qui y est confronté.

\paragraph*{La planification des sprints}
Dans la méthode Scrum, la planification a lieu en équipe, avec le "\textit{product owner}"\footnote{product owner (propriétaire du produit): Il s'agit de la personne qui fait le lien entre la maitrise d'ouvrage et la maitrise d'oeuvre.}. \\
Durant cette réunion, le product owner doit apporter le "\textit{product backlog}"\footnote{product backlog (carnet de produit): il contient la liste ordonnée des besoins et fonctionnalités à implémenter dans la suite du projet.} et décider avec l'équipe quels seront les objectifs du sprint.\\
Durant la réunion, certaines "\textit{user-stories}" du product backlog sont choisies et découpées en tâches détaillées. A l'issue de la réunion, l'équipe repars avec le "\textit{sprint backlog}". Le \textit{sprint backlog} contient la liste des \textit{user-stories} que l'équipe s'est engagé à livrer et la liste des tâches à réaliser pour y parvenir.\\

Du fait de l'activité d'édition logiciel, sur le projet, la planification du sprint est discutée entre le directeur de l'entreprise et le chef de projet. Par la suite, le chef de projet établis le \textit{sprint backlog}, et le transmet à l'équipe (via \textit{Jira}).

\paragraph*{La réunion de présentation du sprint}
Dans la méthode Scrum, il s'agit du \textit{Sprint Review Meeting}. Le but de la réunion est de présenter le travail réalisé pendant le sprint. 

Dans le cadre du projet, cette réunion sert à faire la démonstration au client des fonctionnalités implémentées et à aborder la suite du projet.\\
C'est également l'occasion pour le client de faire un retour sur le travail réalisé.

%TODO OK Méthode de travail employée : méthode de développement, mode de travail en équipe, réunions-présentations etc.

\section{Évaluation du coût du projet réalisé}
Le coût du projet réalisé durant le stage est à ce jour (25 Juillet) de 90 Jour-Hommes.
%TODO Faites une évaluation du coût du projet que vous avez réalisé (en environné), si besoin faites vous aider par votre tuteur en entreprise. 