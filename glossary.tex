\makenoidxglossaries
\defglsentryfmt[main]{\glsgenentryfmt{}{}{*}}

\newglossaryentry{ARL}{
	name=ARL,
	description={est l'acronyme de "Accusé de Réception Logique".}
}

\newglossaryentry{CPS}{
	name=CPS,
	description={est l'acronyme de "Cartes Professionnelles de Santé".}
}

\newglossaryentry{Sesam Vitale}{
  name=Sesam Vitale,
  description={Le programme SESAM-Vitale est un programme de dématérialisation des feuilles de soins pour l'assurance maladie en France, qui repose sur la carte Vitale.}
}

\newglossaryentry{Topaze Maestro}{
  name=Topaze Maestro,
  description={est l'une des versions récentes du progiciel d'e-santé actuellement commercialisé par Supralog. Le logiciel fonctionne comme une application de bureau pour Windows.}
}

\newglossaryentry{Topaze Web}{
  name=Topaze Web,
  description={est la refonte web de Topaze Maestro.}
}

\newglossaryentry{SAAS}{
  name=SAAS,
  description={"Software As A Service" ou logiciel en tant que service. Il s'agit d'un business modèle visant à vendre des accès à une application qui tourne sur un serveur, plutôt que de vendre le logiciel comme un produit.}
}

\newglossaryentry{IDLOG}{
  name=IDLOG,
  description={est un groupe familial créé en 2002 afin de regrouper les sociétés SUPRALOG et IDEA.}
}

\newglossaryentry{IDEA}{
  name=IDEA,
  description={est la société créée en 1987 qui est responsable de la commercialisation de Topaze.}
}

\newglossaryentry{SUPRALOG}{
  name=SUPRALOG,
  description={est la société fondée en 1997 qui a créé le progiciel d'e-santé Topaze.}
}


\newglossaryentry{OLE}{
  name=OLE,
  description={est l'acronyme de "Object Linking and Embedding". Il s'agit d'un protocole développé par Microsoft permettant de faire de la liaison dynamique d'objets dans Windows.}
}

\newglossaryentry{RTF}{
  name=RTF,
  description={est l'acronyme de "Rich Text Format". Il s'agit d'un format de fichier développé par Microsoft et utilisable dans la plupart des éditeurs de textes.}
}

\newglossaryentry{FSE}{
  name=FSE,
  description={est l'acronyme de "Feuille de Soins Électronique". Il s'agit du document envoyé à la caisse, pour remboursement.}
}

\newglossaryentry{DRE}{
	name=DRE,
	description={est l'acronyme de "Demande de Remboursement Électronique". Il s'agit du document envoyé à la complémentaire, pour remboursement.}
}

\newglossaryentry{DAP}{
	name=DAP,
	description={est l'acronyme de "Demande d'Accord Préalable". Les DAP concernent les rééducations en masso-kinésithérapie. Pour chaque type de rééducations, il est prévu qu'un certain nombre de séances soient remboursées par la sécurité sociale. Si ce nombre de séances doit être dépassé, il convient de faire une demande préalable (la DAP).}
}

\newglossaryentry{DSI}{
	name=DSI,
	description={est l'acronyme de "Démarche de Soin Infirmier".}
}

\newglossaryentry{ORM}{
  name=ORM,
  description={est l'acronyme de "Object-relational mapping". Il s'agit d'une technique permettant de manipuler les données de la base de données comme s'il s'agissait d'objets.}
}

\newglossaryentry{SCOR}{
	name=SCOR,
	description={est l'acronyme de "SCannérisation des ORdonnances".}
}

\newglossaryentry{WYSIWYG}{
  name=WYSIWYG,
  description={est l'acronyme de "What You See Is What You Get". Cela désigne le plus souvent un éditeur graphique qui produit le code de ce qui est affiché à l'écran.}
}

\newglossaryentry{JSON}{
	name=JSON,
	description={est l'acronyme de "JavaScript Object Notation". Il s'agit du format utilisé en JavaScript pour décrire des objets.}
}

\newglossaryentry{Spring}{
	name=SPRING,
	description={est un framework applicatif et un conteneur léger pour Java.}
}

\newglossaryentry{IOC}{
	name=IOC,
	description={est l'abréviation de "Inversion Of Control". Spring IOC est un conteneur léger qui implémente le design pattern Inversion de contrôle, qui permet de gérer l'injection des dépendances et leur cycle de vie.}
}

\newglossaryentry{AOP}{
	name=AOP,
	description={est l'abréviation de "Aspect Oriented Programming". Il s'agit d'un nouveau paradigme de programmation qui permet d'injecter dynamiquement des fonctionnalités transversales.}
}

\newglossaryentry{MVC}{
	name=MVC,
	description={est l'abréviation de "Model View Controller". Spring MVC est l'implémentation du patron de conception MVC pour Spring.}
}

\newglossaryentry{HTML5}{
	name=HTML5,
	description={est l'abréviation de "HyperText Markup Language". HTML5 est la 5eme version du langage. Elle apporte de nombreuses fonctionnalités (multimédia, canvas, attributs data-, contenu éditable, etc.).}
}

\newglossaryentry{CSS3}{
	name=CSS3,
	description={est l'abréviation de "Cascading Style Sheets". La troisième version du langage apporte de nombreuses fonctionnalités telles que les animations, les transitions, les media-queries, les dégradés, etc.}
}

\newglossaryentry{Java EE}{
	name=Java EE,
	description={est l'abréviation de "Model View Controller". Spring MVC .}
}

\newglossaryentry{JSF}{
	name=JSF,
	description={est l'abréviation de "JavaServer Faces". Il s'agit d'un framework java permettant de développer des interfaces web à partir de composants.}
}

\newglossaryentry{DTO}{
	name=DTO,
	description={est l'abréviation de "Data Transfer Object". Ce sont des objets utilisés pour transférer les données. Ils sont retournés par l'API REST.}
}

\newglossaryentry{Java}{
	name=Java,
	description={est le langage de programmation orienté objet créé par Sun Microsystems et désormais détenu par Oracle.}
}

\newglossaryentry{Java 8}{
	name=Java 8,
	description={est la version 8 du langage Java. Cette nouvelle version apporte les Streams, les Optional, les interfaces fonctionnelles, les lambda expressions, une nouvelle API pour les dates etc.}
}

\newglossaryentry{JavaScript}{
	name=JavaScript,
	description={est un langage de programmation interprété par le navigateur, utilisé pour développer des applications web. }
}

\newglossaryentry{Hibernate}{
	name=Hibernate,
	description={est un ORM (Object-Relational Mapper) permettant de gérer la persistance des objets Java en base de données relationnelle.}
}

\newglossaryentry{PostgreSQL}{
	name=PostgreSQL,
	description={est un système de gestion de base de données relationnelle et objet.}
}

\newglossaryentry{MongoDB}{
	name=PostgreSQL,
	description={est un système de gestion de base de données orientées documents.}
}

\newglossaryentry{GridFS}{
	name=GridFS,
	description={est un système permettant de stocker des fichiers sur MongoDB.}
}

\newglossaryentry{Tomcat}{
	name=Tomcat,
	description={est un serveur d'applications Java.}
}
