\makenoidxglossaries
\defglsentryfmt[main]{\glsgenentryfmt{}{}{*}}

\newglossaryentry{Sesam Vitale}{
  name=Sesam Vitale,
  description={Le programme SESAM-Vitale est un programme de dématérialisation des feuilles de soins pour l'assurance maladie en France, qui repose sur la carte Vitale.}
}

\newglossaryentry{Topaze Maestro}{
  name=Topaze Maestro,
  description={est l'une des versions rescente du progiciel d'e-santé actuellement commercialisé par Supralog. Le logiciel fonctionne comme une application de bureau pour Windows.}
}

\newglossaryentry{Topaze Web}{
  name=Topaze Web,
  description={est la refonte web de Topaze Maestro.}
}

\newglossaryentry{SAAS}{
  name=SAAS,
  description={"Software As A Service" ou logiciel en tant que service. Il s'agit d'un business modèle visant à vendre des accés à une application qui tourne sur un serveur, plutôt que de vendre le logiciel comme un produit.}
}

\newglossaryentry{IDLOG}{
  name=IDLOG,
  description={est un groupe familial créé en 2002 afin de regouper les sociétés SUPRALOG et IDEA}
}

\newglossaryentry{IDEA}{
  name=IDEA,
  description={est la société créée en 1987 qui est responsable de la commercialisation de Topaze.}
}

\newglossaryentry{SUPRALOG}{
  name=SUPRALOG,
  description={est la société fondée en 1997 qui a créé le progiciel d'e-santé Topaze.}
}


\newglossaryentry{OLE}{
  name=OLE,
  description={est l'accronyme de "Object Linking and Embedding". Il s'agit d'un protocole développé par Microsoft permettant de faire de la liaison dynamique d'objets dans Windows.}
}

\newglossaryentry{RTF}{
  name=RTF,
  description={est l'accronyme de "Rich Text Format". Il s'agit d'un format de fichier développé par Microsoft et utilisable dans la plupart des éditeurs de textes.}
}

\newglossaryentry{FSE}{
  name=FSE,
  description={est l'accronyme de "Feuille de Soins Electronique".}
}

\newglossaryentry{ORM}{
  name=ORM,
  description={est l'accronyme de "Object-relational mapping". Il s'agit d'une technique permettant de manipuler les données de la base de données comme s'il s'agissait d'objets.}
}

\newglossaryentry{WYSIWYG}{
  name=WYSIWYG,
  description={est l'accronyme de "What You See Is What You Get". Cela désigne le plus souvent un éditeur graphique qui produit le code de ce qui est affiché à l'écran.}
}
