Mise en contexte

Le stage s'ancre dans la réalisation du logiciel d'e-santé Topaze Web qui a pour but de remplacer l'actuel logiciel Topaze Maestro.
Topaze Maestro était un logiciel au départ commercialisé en tant que produit (SAB: Software As Bytes), alors que Topaze Web a vocation a être vendu en tant que service (SAAS: Software as a service).
Cette nouvelle version est donc une refonte totale de topaze Maestro : un nouveau projet commence de zéro.

Le stage débute alors que le projet a déjà un an et demi d'existance et qu'une mise en commercialisation est envisagée peu de temps après la fin du stage (Septembre - Octobre). Le but de la commercialisation sera de fournir une version du logiciel contenant le fonctionnalités les plus essentielles de Topaze Maestro. 

Dans ce contexte temporel précis, le but du stage est d'apporter à Topaze Web la fonctionnalité "Dossier Médical du Patient" et de préparer la mise en production.
Le dossier médical d'un patient est une fonctionnalité "composite" qui est constituée de plusieurs modules. La liste des fonctionnalités qui s'y rattachent est détaillée dans la section **********.



Dossier médical : 
	- Liste des documents
	- Voir document
		- Accés aux documents (aspect sécu = token documents)
	- Documents textes
		- Modèles de document
		- Placeholders
		- Gestionnaire de fichiers / Librairie d'images
	- Documents audio
		- Web RTC
		- Choix librairie front-end
		- Intégration
	- Ajout scans d'ordonnance
		- Apport au driver (devices/postes utilisateurs)
		- Fonctionnalité de numérisation
		- Traitement d'image (cropping/rotation/zoom)
	- Installeur Windows pour le driver		




\subsubsection{La bibliothèque d'images}
Dans les documents rich-text, l'utilisateur doit pouvoir insérer une image provenant de son ordinateur et également pouvoir l'enreg


\subsubsection{Les "placeholders"}
Dans l'éditeur de texte de Topaze Maestro, une fonctionnalité permet d'utiliser des emplacements paramétrés (placeholders) qui sont remplacés à la sauvegarde par leur valeur réelle (en base de donnée). \\

Pour utiliser la fonctionnalité, l'utilisateur sélectionne dans une arborescence l'information dont il a besoin (par exemple: le numéro de sécurité sociale du patient) et le logiciel insère dans le texte la balise correspondante (dans notre cas : \textit{[patient.numero_securité_sociale]}). Ensuite, lorsque l'utilisateur clique sur "prévisualiser" ou "sauvegarder", l'emplacement est remplacé par sa valeur réelle.

Cette fonctionnalité permet donc à l'utilisateur d'utiliser dirèctement une partie des données présentes en base de données. 


\subsubsection{Les modèles de texte}
La fonctionnalité des modèles de texte repose sur le fait de pouvoir sauvegarder dans une hiérarchie de dossiers, des documents RTF pouvant contenir des emplacements paramétrés.



	

\\




Plusieurs niveaux séparent le nom d'une balise et l'information de la base 

%TODO 
%BDD                ORM                       Objet Métier           nom du placeholder 
%																	  vu par l'utilisateur
%							   <- mappers ->
%Prescription
%date          -- Prescription.date -- PrescriptionBasicView.date -- ordonnance_dates

La base de données a été insérée dans le schéma dans un souci de complétude, mais elle peut être omise car c'est Hibernate (l'ORM) qui fait le lien avec elle. \\
Ensuite il y a le lien entre les objets métiers et les objets hibernate. En général, il est assuré manuellement et des mappers permettent de remplir les objets métiers.
Enfin, il y a le lien entre les objets métiers et les placeholders affichés à l'utilisateur. Pour aire ce lien, différentes options étaient possibles : 
\begin{itemize}
\item Créer une table qui associe chaque placeholder à 
\end{itemize}
