\newpage
\section{Introduction} % (2-3 ¶) Problématique + objectifs + plan  

Ce rapport fait état du travail réalisé durant mon stage de six mois chez \textit{Supralog}.\\

Historiquement, \textit{Supralog} est la société qui développe le logiciel \textit{\gls{Topaze Maestro}} pour le compte de la société \textit{Idea}. \textit{\gls{Topaze Maestro}} est un logiciel d’e-santé destiné aux cabinets médicaux. Le logiciel permet, entre autres, de gérer les feuilles de soins électroniques, la facturation, les patients et leur dossier médical.\\
\textit{Topaze Maestro} est un logiciel de bureau créé en Power Builder, pour Windows. Progressivement, le programme a évolué afin d'intégrer des éléments de réseau. Dans sa version actuelle il consiste en un client léger installable sur la machine client qui communique en RDP, avec une machine de \textit{Supralog}.\\

Afin de suivre les évolutions du marché, \textit{Supralog} souhaite produire une nouvelle version de Topaze qui soit entièrement dématérialisée et commercialisée en tant que service (\gls{SAAS}). Cette nouvelle version qui s’appelle \textit{\gls{Topaze Web}} est une refonte globale du logiciel. Mon travail durant le stage sera de participer à sa conception et à son développement.\\

Dans ce rapport j’évoquerai d’abord le contexte du stage, puis je détaillerai la problématique du projet avant d’aborder la solution mise en place et les aspects liés à la gestion du projet. 

%Enfin je livrerai mes premières impressions sur cette première partie du stage.
