\newpage
\section*{Résumé}  

Dans le cadre de ma dernière année d'école d'ingénieur à l’Ensimag, j’ai réalisé mon projet de fin d’études d’une durée de six mois au sein de l’entreprise Supralog.\\

Le but du projet était de participer à la conception et au développement d’une application web d’e-santé à destination des cabinets médicaux. Cette application réalisée en Java, Spring et JSF selon une architecture 3-Tiers est la refonte d'un logiciel plus ancien qui est distribué en tant qu'application de bureau pour Windows. Le logiciel permet entre autres de gérer les feuilles de soins électroniques, la facturation, l'activité d'un cabinet et utilise la technologie \textit{\gls{Sesam Vitale}}.\\
%legacy

Dans ce cadre, j’ai eu à réaliser le dossier médical d'un patient et les différentes fonctionnalités permettant de le gérer. Pour ce faire, j'ai d'abord fait un état des lieux des fonctionnalités présentes dans le logiciel d'origine, puis j'ai analysé l'existant sur la nouvelle application et j'ai conçu et implémenté une solution qui puisse s'y greffer.\\

Finalement, ce stage m'aura permis de développer mes compétences sur les technologies Java 8, Spring, JSF et Hibernate. J'aurais  travaillé au sein d'une équipe qui applique les méthodes agiles et sur un projet qui utilise l'intégration continue. Enfin, cela m'a donné l'occasion de découvrir les aspects métiers liés aux prescriptions médicales, aux feuilles de soins et à l'assurance maladie.\\

\textbf{Mots-clé}: Java, JSF, Spring, Hibernate, e-santé, Architecture 3-Tiers, Sesam Vitale.